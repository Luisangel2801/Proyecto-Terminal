\setcounter{page}{1} % Iniciar la numeración de las páginas en este punto

\section{Introducción}


\section{Antecedentes}

    \subsubsection*{A chest-based continuous cuffless blood pressure method: Estimation and evaluation using multiple body sensors \cite{bodySensor}.}

    El artículo analiza el desarrollo de un dispositivo no invasivo para la estimación de la presión arterial (PA) utilizando sensores colocados en el pecho utilizando la bioimpedancia (BImp) como alternativa a la fotopletismografía (PPG) para la extracción del tiempo de llegada del pulso (PAT), se tomaron cinco lecturas de tiempo de pulso diferentes y se pudo estimar la PA sistólica y diastólica. 
    
    Se analizaron los datos de 41 participantes en diversas condiciones fisiológicas, incluidos los cambios de postura y los resultados mostraron que la combinación de PAT con la frecuencia cardíaca 
    mejoró la precisión del cálculo de la PA, y las lecturas de PAT basadas en BImp fueron un 3\% más precisas que las basadas en PPG, lo que destaca el potencial de BImp para un control más eficaz de la PA.

    \subsubsection*{An Arterial Compliance Sensor for Cuffless Blood Pressure Estimation Based on Piezoelectric and Optical Signals \cite{piezoelectric}.}

    Este artículo propone el desarrollo de un pequeño sistema de monitoreo que integra una matriz de sensosres piezoeléctricos y un sensor óptico que monitorea las señales fisiológicas de la arteria radial. El sistema hace el cálculo del tiempo de tránsito del pulso (PTT) y correlaciona la ecuación de Moens-Korteweg con la velocidad de onda de pulso (PWV) para estimar la presión arterial sistólica (PAS) y diastólica (PAD).

    En un experimento con 20 participantes se compararon dos métodos de estimación de la presión arterial, el primero utilizando el modelo de regresión y el segundo modelo $P-\beta$ basado en la ecuación de Moens-Korteweg.

    \subsubsection*{Development of IoT Based Cuffless Blood Pressure Measurement System \cite{Norsuriati_2021}.}

    El artículo presenta el desarrollo de un sistema IoT de medición de presión arterial sin uso de un baumanómetro, se propone un método basado en el tiempo de tránsito del pulso (PTT). Este método correlaciona el tiempo de retraso entre las señales fotopletismografía (PPG) registrada en la punta del dedo y el lóbulo de la oreja.

    Los datos son recolectados mediante un microcontrolador Arduino Uno y procesados con el software MATLAB para eliminar el ruido y obtener los picos de la señal PPG para calcular el PTT.
    
    Los resultados son mostrados en la aplicación ThinkSpeak y ThingView debido a que tiene la capacidad de almacenar y visualizar los datos en tiempo real. El error medio y la deviación estándar para la presión arterial sistólica (PAS) estimada es de $22,5 \pm 20,6$ mmHg y para la presión arterial diastólica (PAD) es de $1,6 \pm 1,2$ mmHg.

    \subsubsection*{Diseño de un sistema internet de las cosas (IoT) para el monitoreo de la presión arterial \cite{Estrada_2021}. }

    En este artículo se presenta los procesos de diseño y construcción de un prototipo biomédico IoT para el monitoreo de la presión arterial de pacientes en su lugar de residencia. El sistema consta en colocar un brazalete al paciente a la altura del corazón, el cual se le conecta una bomba de aire el cual infla el brazalete y un sensor de presión diferencial MPX5050DP, se realiza una conversión análoga-digital con ayuda de un microcontrolador y se envía los datos meidante una API REST al servidor web IoT ThinkSpeak donde se almacenan y se pueden visualizar en tiempo real.

    \subsubsection*{Development of Real-Time Cuffless Blood Pressure Measurement Systems with ECG Electrodes and a Microphone Using Pulse Transit Time (PTT) \cite{Electrodes_Microphone}.}

    El estudio habla sobre el desarrollo de un sistema de medición de la presión arterial en tiempo real sin uso de un baumanómetro, utilizando electrodos de ECG y un micrófono en lugar de un sensor de fotopletismografía (PPG). El sistema mide la onda de pulso sanguíneo en la arteria radial de la muñeca, calculando la presión arterial sistólica (SBP) y diastólica (DBP) mediante el tiempo de tránsito del pulso (PTT) entre el pico R del ECG y puntos característicos de la onda de pulso.
    
    Las estimaciones de SBP y DBP fueron comparables a las de un monitor de presión arterial comercial, con un error absoluto medio (MAE) de $2.72 \pm 3.42$ mmHg para SBP y $2.29 \pm 3.53$ mmHg para DBP.


\newpage
\section{Justificación}

En México, las enfermedades cardiovasculares representan la primera causa de muerte, siendo responsable de más de 220 mil fallecimientos en 2021 \cite{SSFallecimientos}. Entre los factores de riesgo más importantes para estas patologías se encuentra la hipertensión arterial, conocida también como el ``asesino silencioso", ya que no presenta síntomas y puede pasar desapercibida durante años, hasta que desarrollan complicaciones graves como infartos o accidentes cerebrovasculares. Según la Encuesta Nacional de Salud y Nutrición (ENSANUT), cerca del 30\% de la población adulta en México padece hipertensión arterial \cite{ENSANUT}, pero una gran proporción de ellos no recibe un tratamiento adecuado o no se diagnostica a tiempo debido a la falta de acceso a un monitoreo continuo y preciso.

Los métodos para la medición de la presión arterial, como el baumanómetro, son invasivos y requieren de personal capacitado, lo que limita su uso en el monitoreo continuo de la presión arterial. Esta limitación es crítica en un país como México, donde el sistema de salud puede ser limitado, especialmente en zonas rurales y marginadas. Por lo tanto, es necesario desarrollar dispositivos no invasivos y de bajo coste que permitan a las personas el monitorear su presión arterial de forma continua y precisa, para detectar y tratar a tiempo la hipertensión arterial y de esta forma prevenir complicaciones graves.

Este proyecto tiene como objetivo desarrollar un sistema basado en sensores para la estimación de la presión arterial de forma no invasiva, que permita a las personas monitorear su presión arterial de forma continua y precisa, y enviar los datos a un dispositivo móvil para la visualización y el almacenamiento de los registros para llevar un control de su salud cardiovascular. Facilitando así la detección temprana de la hipertensión arterial y la prevención de complicaciones graves asociadas a esta enfermedad.

El desarrollo de un sistema de monitoreo no invasivo de la presión arterial permitirá a las personas tener un mayor control de su salud cardiovascular, detectar a tiempo la hipertensión arterial y prevenir complicaciones graves asociadas a esta enfermedad, mejorando así su calidad de vida y reduciendo la carga de enfermedades cardiovasculares en México.


\newpage
\section{Objetivos}
    \subsection{Objetivo General}
    Desarrollar un dispositivo para estimar la presión arterial de forma no invasiva utilizando sensores.
    \subsection{Objetivos Específicos}
    \begin{itemize}
        \item Diseñar un dispositivo utilizando sensores para la obtención de datos fisiológicos.
        \item Enviar los datos mediante Bluetooth a un dispositivo móvil.
        \item Implementar un algoritmo de procesamiento de señales para la estimación de la presión arterial.
        \item Visualizar los datos obtenidos en una aplicación android.
    \end{itemize}

\newpage
\section{Marco Teórico}

    \subsection{El sistema de conducción del corazón}

    El corazón es un músculo que late y bombea continuamente sangre al resto del cuerpo. Lo que comúnmente llamamos latido cardíaco es en realidad la contracción rítmica de las cuatro cavidades del corazón. Cada latido cardíaco es estimulado por señales eléctricas que viajan a través de una vía específica del corazón. Estas señales se pueden registrar mediante un electrocardiograma (ECG).

     Este proceso comienza en el nódulo senoauricular (nódulo SA), situado en la aurícula derecha. La señal eléctrica se propaga a las aurículas, provocando su contracción y el empuje de sangre hacia los ventrículos. Luego, la señal llega al nódulo auriculoventricular (nódulo AV) y se desplaza a través del haz de His, que se divide en ramas izquierda y derecha dentro de los ventrículos. Finalmente, la señal viaja por las fibras de Purkinje, que son fibras musculares especializadas que se encuentran en las paredes de los ventrículos, el cual provoca la contracción de los ventrículos, bombeando sangre a los pulmones y al resto del cuerpo. Este sistema actúa como el marcapasos natural del cuerpo, manteniendo un ritmo cardíaco normal de 60 a 100 latidos por minuto. Alteraciones en este sistema pueden resultar en ritmos cardíacos anormales y afectar el flujo sanguíneo del cerebro y otras partes del cuerpo \cite{SistemaConduccionMSD}.

    \begin{figure}[H]
        \centering
        \includegraphics[width=0.8\textwidth]{img/sistemaConduccion.png}
        \caption[Sistema de conducción cardíaca y tiempos de conducción de los respectivos segmentos]{Sistema de conducción cardíaca y tiempos de conducción de los respectivos segmentos\footnotemark}
        \label{fig:sistemaConduccion}
    \end{figure}
    \footnotetext{Localización de las células marcapasos dentro del sistema de conducción del corazón. Imagen tomada de lecturio.com. Fuente: \url{https://goo.su/MSZMxUj}}

    \subsection{Electrocardiograma (ECG)}
    Un electrocardiograma (ECG) es un estudio que registra el voltaje generado por los vectores de despolarización y repolarización de las células cardiacas en relación con el tiempo, es una herramienta útil para evaluar la función cardíaca y detectar problemas cardíacos \cite{ECG_Definicion}.

    Es una prueba no invasiva e indolora, que se realiza para detectar problemas cardíacos o controlar el estado del corazon. El ECG generalmente utiliza sensores (electrodos), colocados sobre la piel del pecho, pueden detectar señales eléctricas del corazón. Las señales de estos sensores se conectan a circuitos electrónicos simples con amplificadores, filtros y convertidores analógico-digitales, que registran la señal eléctrica y la muestran en un monitor o la imprimen para su posterior análisis.

    Las ondas del ECG son las siguientes:

    \begin{itemize}
        \item \textbf{Onda P}: Representa la despolarización auricular, es la suma de los vectores de despolarización de ambas aurículas.
        \item \textbf{Intervalo PR}: Representa el tiempo transcurrido desde la despolarización auricular, hasta la despolarización ventricular. Incluye la onda P y el segmento PR. Éste último elemento es una línea isoeléctrica, establecida gracias al retardo fisiológico que sufre la conducción eléctrica en el nodo aurículoventricular. Sin este retraso mencionado, las aurículas y los ventrículos se despolarizarían casi al mismo tiempo, siendo imposible el funcionamiento correcto del corazón para que la sangre pase por sus diferentes cavidades ordenadamente.
        \item \textbf{Onda Q}: Indica el inicio de la despolarización ventricular, específicamente el vector de despolarización septal.
        \item \textbf{Onda R}: Representa el segundo vector de despolarización, correspondiente a la pared libre del ventrículo izquierdo. Es normalmente la onda con mayor voltaje, debido a que el ventrículo izquierdo es el que mayor cantidad de células posee, por ende, la actividad eléctrica es mayor y el vector es más grande.
        \item \textbf{Onda S}: Corresponde al último vector de despolarización ventricular, originado en las bases de los ventrículos.
        \item \textbf{Complejo QRS}: Es la suma de los tres vectores de despolarización anteriores, y juntos representan a la despolarización ventricular.
        \item \textbf{Segmento ST}: Es un periodo de inactividad que separa la despolarización ventricular de la repolarización ventricular, va desde el final del complejo QRS hasta el comienzo de la onda T. 
        \item \textbf{Intervalo QT}: Se extiende desde el comienzo del complejo QRS hasta el final de la onda T y representa la sístole eléctrica ventricular, o lo que es lo mismo, el conjunto de la despolarización y repolarización ventricular. La medida de este intervalo depende de la frecuencia cardiaca, de forma que el intervalo QT se acorta cuando la frecuencia cardiaca es alta, y se alarga cuando la frecuencia cardiaca es baja. Por lo anterior, cuando se mide, es necesario corregirlo de acuerdo con la frecuencia cardíaca utilizando la fórmula de Bazett
        \item \textbf{Onda T}: Es la onda que representa la repolarización ventricular.
        \item \textbf{Onda U}: Es una onda de escaso voltaje que puede o no estar presente en el trazado del electrocardiograma. Se debe a la repolarización de los músculos papilares
        \item \textbf{Intervalo RR}: Es el intervalo que abarca desde una onda R, hasta la onda R de la siguiente despolarización, es decir dos ondas R sucesivas. En un paciente sano, debe permanecer a un ritmo constante. La medida de este intervalo dependerá de la frecuencia cardiaca.
    \end{itemize}

    \begin{figure}[H]
        \centering
        \includegraphics[width=0.8\textwidth]{img/ECG_ondas.png}
        \caption[Ciclo completo de un electrocardiograma]{Ciclo completo de un electrocardiograma\footnotemark}
        \label{fig:ECG_ondas}
    \end{figure}
    \footnotetext{Muestra el ciclo completo de un electrocardiograma, con las ondas P, Q, R, S, T y U, y los intervalos PR, QRS, ST y QT. Imagen tomada de encyclopedia.pub. Fuente: \url{https://encyclopedia.pub/entry/52174}}

    El ECG tiene las características de baja frecuencia y concentración de energía, y la señal es débil, fácilmente perturbada por el ruido, como interferencias eléctricas, campos electromagnéticos externos, movimiento de electrodos (ruido instantáneo)  y el desplazamiento de la línea de base que es causado principalmente por la respiración. Estos ruidos dificultan el diagnóstico médico y pueden provocar errores en la identificación de enfermedades. Por ello, es esencial aplicar técnicas adecuadas para eliminar el ruido y mejorar la calidad de los datos de ECG muestreados \cite{AlMahamdy_2014}.

    \subsection{Fotopletismografía (PPG)}
        \subsubsection{Principios de la fotopletismografía}
            La fotopletismografía (PPG, por sus siglas en inglés) es una técnica óptica que puede utilizarse para detectar cambios en el volumen sanguíneo \cite{Hertzman_1938}. Un PPG es un dispositivo simple que consta de una fuente de luz y un detector; se han desarrollado dispositivos PPG que utilizan luz de diferentes longitudes de onda e intensidades basadas en la tecnología de diodos emisores de luz (LED). La cantidad de energía transferida a la piel por la luz depende de su longitud de onda; por ejemplo, la luz verde se utiliza con frecuencia y tiene una buena relación señal-ruido \cite{Challoner_1979}. Además, las señales de luz roja, verde y azul (RGB) permiten determinar el pulso y las frecuencias respiratorias.

            La función del fotodetector es detectar y cuantificar la luz absorbida durante el flujo pulsátil y no pulsátil. Durante el flujo pulsátil, la luz se absorbe por el cambio en el flujo sanguíneo dentro de las arterias, que es sincrónico con un latido del corazón. Durante el flujo no pulsátil, la luz se absorbe por los tejidos de fondo. Por lo tanto, un fotodetector detecta el cambio volumétrico en el flujo sanguíneo en las arterias al detectar la diferencia de intensidad de luz. La medición de este cambio en la intensidad de la luz ayuda a analizar la funcionalidad del corazón.

            En los últimos treinta años, el número de artículos publicados sobre PPG ha aumentado significativamente, abarcando tanto la investigación básica como la aplicada. En todas estas publicaciones, la PPG ha sido elogiada como una técnica óptica no invasiva, de bajo costo y simple para medir parámetros fisiológicos aplicados en la superficie de la piel. \cite{PPG}.

            La popularidad de este tema se puede atribuir a la comprensión de que la PPG tiene implicaciones importantes para una amplia gama de aplicaciones. Entre muchas, ayuda en la detección de oxígeno en sangre, la evaluación cardiovascular y el control de los signos vitales. Además, la importante contribución de la PPG en los dispositivos portátiles ha elevado exponencialmente la popularidad y la facilidad de uso de la PPG \cite{allen_2007}.

        \subsubsection{Modos de medición}
            La tecnología PPG mide los cambios en el volumen de sangre en los tejidos durante un ciclo cardíaco mediante una fuente de luz. Esta medición volumétrica proporciona información importante sobre el sistema cardiovascular. Un sensor PPG consta principalmente de dos componentes electrónicos: un emisor de luz y un detector de intensidad lumínica.

            Generalmente, se utiliza un LED como emisor de luz y un fotodetector para captar los cambios en la intensidad lumínica. Un pulso PPG correspondiente a un latido del corazón que incluye las fases sistólica y diastólica. Durante la fase sistólica, el corazón se contrae y empuja la sangre rica en oxígeno hacia los tejidos y órganos, lo que incrementa el volumen de sangre en las arterias. Esto provoca que las células sanguíneas absorban más luz, por lo que la cantidad de luz detectada por el fotodetector es menor. En cambio, durante la fase diastólica, el corazón se relaja y la sangre regresa a él, disminuyendo el volumen sanguíneo en las arterias, como resultado, se absorbe menos luz y el fotodetector registra un aumento en la intensidad lumínica.\cite{Hiiberia_2023}.

            Dependiendo de la aplicación y la ubicación del sensor, el PPG se puede usar en modo transmisivo o en modo de reflexión.

            \begin{enumerate}
                \item Cuando un fotodetector y un LED se colocan en lados opuestos de un dedo para detectar la luz transmitida, este arreglo se conoce como modo transmisivo. En este modo, la sonda está configurada de manera que el fotodetector y el LED se enfrentan con una capa de tejido entre ellos. La detección en modo transmisivo depende de la luz que atraviesa las partes del cuerpo, por lo que se prefieren estructuras delgadas como el lóbulo de la oreja y el dedo.
                \item Cuando el fotodetector y el LED se colocan en el mismo lado de un dedo para detectar la luz reflejada, se utiliza el modo reflectivo. En este arreglo, ambos sensores se sitúan uno al lado del otro con una pequeña separación. Por lo tanto, el modo reflectivo puede emplearse en cualquier parte del cuerpo, como la frente o la muñeca.
            \end{enumerate}

            \begin{figure}[H]
                \centering
                \begin{subfigure}[b]{0.45\linewidth}
                    \includegraphics[width=\linewidth]{img/PPG_transmisivo.png}
                    \caption{Sensor PPG en modo de transmisión}
                    \label{fig:PPG_transmisivo}
                \end{subfigure}
                \begin{subfigure}[b]{0.45\linewidth}
                    \includegraphics[width=\linewidth]{img/PPG_reflexion.png}
                    \caption{Sensor PPG en modo de reflexión}
                    \label{fig:PPG_reflexion}
                \end{subfigure}
                \caption[Modos de medición de la fotopletismografía]{Modos de medición de la fotopletismografía\footnotemark}
                \label{fig:modosMedicionPPG}
            \end{figure}
            \footnotetext{Muestra la posición de los sensores PPG en modo de transmisión y en modo de reflexión, los dos modo se utiliza para medir la fotopletismografía. Imagen tomada de aiva.hi. Fuente: \url{https://goo.su/pg8UGx}}

            \subsubsection{Forma de onda de un PPG}

            En la figura \ref{fig:PPG} se muestra la señal característica de un fotopletismógrado, la cual está directamente relacionada con la frecuencia cardíaca, donde cada periodo de la señal corresponde a una pulsación del corazon.

            La señal representa dos picos por cada periodo, el pico mayor representa la presión sistólica, y el segundo pico representa el inicio de la presión diastólica cuyo valor es el mínimo de la curva \cite{Celi_2011}.

            Para calcular la frecuencia cardíaca (FC) a partir de la señal PPG, se utiliza la ecuación~\ref{eq:FrecienciaCardiaca}, donde el intervalo entre pico a pico es el tiempo transcurrido entre dos pulsaciones del corazón.

            \begin{equation}
                \label{eq:FrecienciaCardiaca}
                FC = \frac{60}{\textit{Intervalo entre picos}}
            \end{equation}

            \begin{figure}[H]
                \centering
                \includegraphics[width=0.7\textwidth]{img/PPG_senial.png}
                \caption[Modelo matemático de una señal pura de una fotopletismografía]{Modelo matemático de una señal pura de una fotopletismografía\footnotemark}
                \label{fig:PPG}
            \end{figure}
            \footnotetext{Muestra una representación de una señal pura de una fotopletismografía, donde se observa la forma de onda de la señal con una frecuencia de 1.5 Hz, que representa 90 latidos por minuto. Imagen tomada del articulo ´´Body sensor network for mobile health monitoring, a diagnosis and anticipating system´´. Fuente: \url{https://doi.org/10.1109/jsen.2015.2464773}}

        
        \subsubsection{Componente AC y DC de la señal PPG}
            La figura \ref{fig:PPG_AC_DC} muestra un ejemplo de una forma de onda fotopletismográfica, que tiene componentes de corriente continua (DC) y corriente alterna (AC). El componente DC de la forma de onda PPG corresponde a la señal óptica transmitida o reflejada del tejido y depende de la estructura del tejido y del volumen promedio de la sangre arterial y venosa. El componente DC cambia lentamente con la respiración, mientras que el componente AC fluctúa de acuerdo con los cambios en el volumen sanguíneo que ocurren entre las fases sistólica y diastólica del ciclo cardíaco \cite{Tamura_2019}.

            Cuando el corazón bombea sangre durante la sístole, el aumento del volumen sanguíneo en los tejidos periféricos provoca una mayor absorción de la luz o una menor reflexión, lo que da lugar a una desviación hacia abajo en la forma de onda PPG. Durante la diástole, cuando el corazón se relaja, el volumen sanguíneo disminuye, lo que da lugar a una desviación hacia arriba en la forma de onda PPG. La tecnología basada en PPG calcula las mediciones en función de los cambios reflejados en la forma de onda \cite{Cabessa_2024}.

            \begin{figure}[H]
                \centering
                \begin{subfigure}[b]{0.45\linewidth}
                    \includegraphics[width=\linewidth]{img/PPG_AC_DC.png}
                \end{subfigure}
                \begin{subfigure}[b]{0.45\linewidth}
                    \includegraphics[width=\linewidth]{img/PPG_Componente_AC_DC.png}
                \end{subfigure}
                \caption[Componente AC y DC de la señal PPG]{Componente AC y DC de la señal PPG\footnotemark}
                \label{fig:PPG_AC_DC}
            \end{figure}
            \footnotetext{Muestra un ejemplo de forma de onda de la señal PPG, donde se observa el componente AC y DC de la señal. Imagen tomada del artículo ``Wearable Photoplethysmographic Sensors—Past and Present". Fuente: \url{https://doi.org/10.3390/electronics3020282}}

        \subsubsection{Longitud de onda de la luz emitida}
            La piel del cuerpo está compuesta principalmente por tres capas de tejido: la epidermis, la dermis y la hipodermis. Debido a la absorción, solo las ondas de luz con una longitud de onda mayor pueden penetrar a través de todas ellas. La hemoglobina oxigenada absorbe la luz infrarrojo (NIR), mientras que la hemoglobina desoxigenada absorbe luz en la longitud de onda roja. Como resultado, los PPG que emplean LED y fotodetectores con longitudes de onda NIR y roja se utilizan comúnmente en el control clínico para calcular la concentración de hemoglobina.

            Sin embargo, el movimiento del paciente influye en la precisión de la medición y está relacionado con la longitud de onda de la luz utilizada. La luz de longitud de onda más larga, como el infrarrojo, se ve más afectada por el movimiento debido a que penetra profundamente en el tejido. Por otro lado, la luz de longitud de onda más corta (luz verde) generalmente está libre de los efectos del movimiento, ya que penetra menos en el tejido corporal. Por lo tanto, para mitigar el impacto del movimiento y la absorción de luz por los tejidos, se ha propuesto el uso de PPG basados en sensores ópticos de múltiples longitudes de onda para detectar variaciones en el flujo sanguíneo a diferentes profundidades de la piel. \cite{Hiiberia_2023}.

            \begin{figure}[H]
                \centering
                \includegraphics[width=0.8\textwidth]{img/PPG_luz.jpg}
                \caption[Penetración de las longitudes de onda de la luz en la piel]{Penetración de las longitudes de onda de la luz en la piel\footnotemark}
                \label{fig:PPG_longitud_onda}
            \end{figure}
            \footnotetext{Muestra la penetración de la luz en la piel, donde se observa que la luz de longitud de onda más larga (infrarroja) penetra más en la piel que la luz de longitud de onda más corta (Azul). Imagen tomada de Journal of Clinical Monitoring and Computing. Fuente: \url{https://goo.su/jnUP8yi}}

            Dentro de la región visible, el pico de absorción dominante está en la región azul del espectro, seguido de la región verde-amarilla (500-600 nm), correspondiente a los glóbulos rojos. La luz de longitudes de onda más cortas es absorbida fuertemente por la melanina. El agua absorbe la luz en las regiones ultravioleta e infrarroja (IR) más largas. La luz roja (660 nm) e IR (940nm) pasa a través del tejido y la sangre. Por lo tanto, la luz IR se ha utilizado en sensores PPG.
        
            En la última década, la eficiencia de los LED ha aumentado y su voltaje directo ha disminuido, lo que resulta en un mayor número de lúmenes por watt. Gracias a la iluminación de alta potencia, la variación en el ciclo cardíaco entre las fases sistólica y diastólica es más pronunciada en la longitud de onda verde. Por ello, la luz verde se ha empleado en sensores PPG para medir la frecuencia cardíaca y la saturación de oxígeno en la sangre. \cite{Tamura_2019}.

    \subsection{Presión Arterial}
        La presión arterial es la fuerza que ejerce la sangre contra las paredes de las arterias. La presión arterial se mide en milímetros de mercurio (mmHg) e incluye dos mediciones: la presión sistólica, que se mide durante el latido del corazón (momento de presión máxima), y la presión diastólica, que se mide durante el descanso entre dos latidos (momento de presión mínima) \cite{PresionArterialDefinicion}.

        La presión arterial normal es de 120/80 mmHg, la presión arterial alta (hipertensión) es de 140/90 mmHg o más y la presión arterial baja (hipotensión) es de 90/60 mmHg o menos \cite{DOF}.

    \subsection{Amplificadores}
        La mayoría de las señales bioeléctricas del cuerpo humano son señales con una magnitud del orden de máximo $5 mV_{pp}$ y para poder ser registradas y analizadas, es necesario amplificarlas.

        \subsubsection{Amplificador operacional}
            Los amplificadores operacionales son dispositivos electrónicos que se utilizan para amplificar señales eléctricas. Estos dispositivos tienen dos entradas, una inversora y otra no inversora, y una salida; La salida del amplificador operacional es proporcional a la diferencia de voltaje entre las dos entradas multiplicada por un factor de ganancia. La ganancia de un amplificador operacional se puede ajustar mediante la selección de resistencias externas.

            \begin{figure}[H]
                \centering
                \includegraphics[width=0.5\textwidth]{img/Desarrollo/Amplificador_Operacional.png}
                \caption[Terminales de entrada y salida de un amplificador operacional]{Terminales de entrada y salida de un amplificador operacional\footnotemark}
                \label{fig:Amplificador_Operacional}
            \end{figure}
            \footnotetext{Muestra el símbolo electrónico de un amplificador operacional, con sus dos entradas y su salida.}

            \paragraph{Amplificador Inversor}
                Es una configuración en un amplificador operacional que se caracteriza principalmente porque su entrada, es decir, donde suministraremos un voltaje, se coloca en la terminal inversora del amplificador y la terminal no inversora se conecta a tierra. A la salida del amplificador obtendremos una señal amplificada e inversa (como su nombre lo indica) a su entrada. La ganancia de un amplificador inversor se puede calcular mediante la ecuación~\ref{eq:ganancia_inversor}.

                \begin{equation}
                    \label{eq:ganancia_inversor}
                    G = -\frac{R_f}{R_i}
                \end{equation}

                \begin{figure}[H]
                    \centering
                    \includegraphics[width=0.5\textwidth]{img/Desarrollo/Amplificador_Inversor.png}
                    \caption[Amplificador inversor]{Amplificador inversor\footnotemark}
                    \label{fig:Amplificador_Inversor}
                \end{figure}
                \footnotetext{Muestra el diagrama esquemático de un amplificador inversor, con una ganancia de $-\frac{R_f}{R_i}$.}

                \paragraph{Amplificador No Inversor}
                Este tipo de configuración se caracteriza por tener conectado la señal de entrada (voltaje) a la terminal no inversora, esto nos indica que la ganancia será positiva (al contrario del inversor). También tiene como característica una realimentación negativa (Rf).La ganancia de un amplificador no inversor se puede calcular mediante la ecuación~\ref{eq:ganancia_no_inversor}.

                \begin{equation}
                    \label{eq:ganancia_no_inversor}
                    G = \frac{R_f}{R_i} + 1
                \end{equation}

                \begin{figure}[H]
                    \centering
                    \includegraphics[width=0.5\textwidth]{img/Desarrollo/Amplificador_No_Inversor.png}
                    \caption[Amplificador no inversor]{Amplificador no inversor\footnotemark}
                    \label{fig:Amplificador_No_Inversor}
                \end{figure}
                \footnotetext{Muestra el diagrama esquemático de un amplificador no inversor, con una ganancia de $1 + \frac{R_f}{R_i}$.}

                \paragraph{Amplificador Diferencial}
                El amplificador diferencial es un tipo de amplificador operacional que amplifica la diferencia de voltaje entre dos entradas y suprime la señal común a ambas entradas. La ganancia de un amplificador diferencial se puede calcular mediante la ecuación~\ref{eq:ganancia_diferencial}.

                \begin{equation}
                    \label{eq:voltaje_diferencial}
                    v_0 = 1 + \frac{R2}{R1} \cdot \frac{R3}{R3 + R4} \cdot v2 - \frac{R2}{R1} \cdot v1
                \end{equation}

                Si $R1 = R3$ y $R2 = R4$, la ecuación~\ref{eq:voltaje_diferencial} se simplifica a la ecuación~\ref{eq:voltaje_diferencial_simplificada}.

                \begin{equation}
                    \label{eq:voltaje_diferencial_simplificada}
                    v_0 = \frac{R2}{R1} \cdot (v2 - v1)
                \end{equation}

                \begin{equation}
                    \label{eq:ganancia_diferencial}
                    G = \frac{R2}{R1}
                \end{equation}

                \begin{figure}[H]
                    \centering
                    \includegraphics[width=0.5\textwidth]{img/Desarrollo/Amplificador_Diferencial.png}
                    \caption[Amplificador diferencial]{Amplificador diferencial\footnotemark}
                    \label{fig:Amplificador_Diferencial}
                \end{figure}
                \footnotetext{Muestra el diagrama esquemático de un amplificador diferencial, con una ganancia de $\frac{R2}{R1}$.}

                \paragraph{Amplificador de Instrumentación}
                Para amplificar señales bioeléctricas, como las del ECG, se requieren dos características en un amplificador: la primera es que presente una alta impedancia en sus termminales de entrada (elimina posibles caídas de voltaje de la señal cardiaca que den como resultado la reducción o anulación de su amplitud), y la segunda es que solamente amplifique la diferencia de voltaje existente entre dichas terminales. 
                El amplificador que reúne las dos caracteríasticas mencionadas es el amplificador de instrumentación.

                Para hacer posible lo anterior, todos los amplificadores de instrumentación se basan en el diseño mostrado en la figura~\ref{fig:Amplificador_Instrumentacion} que tiene una etapa con amplificadores no inversores (brindan alta impedancia) y enseguida una etapa de amplificación diferencial (casi siempre unitaria)\cite{Diaz_amplificacion_señales}.

                La expresión para calcular la ganancia y el voltaje de un amplificador de instrumentación se muestra en la ecuación~\ref{eq:ganancia_amplificador_instrumentación}.

                \begin{equation}
                    \label{eq:voltaje_amplificador_instrumentación}
                    v_0 = (1 + \frac{2R1}{RG}) \cdot (v2 - v1)
                \end{equation}

                \begin{equation}
                    \label{eq:ganancia_amplificador_instrumentación}
                    G = 1 + \frac{2R1}{RG}
                \end{equation}

                \begin{figure} [H]
                    \centering
                    \includegraphics[width=0.8\textwidth]{img/Desarrollo/Amplificador_Intrumentacion.png}
                    \caption[Amplificador de Instrumentación]{Amplificador de Instrumentación\footnotemark}
                    \label{fig:Amplificador_Instrumentacion}
                \end{figure}
                \footnotetext{Diagrama de los elementos básicos que conforman las dos etapas en un amplificador de instrumentación. Imagen tomada de \textit{Fundamentals of operational amplifiers and linear integrated circuits}, p. 34.}

        \subsubsection{Amplificador de Intrumentación AD620}
            El amplificador de instrumentación AD620, es un amplificador de precisión de bajo costo y bajo consumo de energía fabricado por Analog Devices, que se utiliza para amplificar señales de baja amplitud y alta impedancia, como las señales ECG. \cite{AD620_AnalogDevices,AD620_DigiKey}.

            \begin{figure}[H]
                \centering
                \includegraphics[width=0.8\textwidth]{img/Desarrollo/AD620.png}
                \caption[Diagrama esquemático del amplificador de instrumentos AD620.]{Diagrama esquemático del amplificador de instrumentos AD620\footnotemark}
                \label{fig:AD620}
            \end{figure}
            \footnotetext{Muestra que el amplificador de instrumentación AD620 es equivalente a un amplificador diferencial conectado a dos amplificadores no inversores. Imagen tomada del artículo \textit{Axial and Lateral Small Strain Measurement of Soils in Compression Test using Local Deformation Transducer} de la revista \textit{Journal of Engineering and Technological Sciences}. Fuente: \url{https://goo.su/fHljmIK}}

            \hfill \break
            Características del AD620:
            \begin{itemize}
                \item \textbf{Ganancia configurable:} Ajustable de 1 a 10,000 mediante una resistencia externa.
                \item \textbf{Bajo Nivel de Ruido:} Minimiza errores en la señal amplificada, asegurando una representación precisa de la actividad eléctrica cardíaca.
                \item \textbf{Baja Corriente de Polarización de Entrada:} Con una tensión de polarización máxima de $50 \mu V$ y una deriva mínima de $0,6 \mu V / ^{\circ} C$.
                \item \textbf{Bajo Consumo de Potencia:} Consume un máximo de $1,3 mA$, adecuado para dispositivos portátiles o alimentados por batería.
                \item \textbf{Rango de Voltaje de Alimentación:} Funciona con voltajes de $\pm 2,3 V$ a $\pm 18 V$.
            \end{itemize}

            El AD620 es una gran alternativa al arreglo tradiccional de tres amplificadores operacionales, ya que ofrece una mayor precisión y estabilidad, y requiere menos componentes externos. 

            \begin{figure}[H]
                \centering
                \includegraphics[width=0.6\textwidth]{img/Desarrollo/Amplificador_AD620_comparacion.png}
                \caption[Comparación de la taza de error entre el AD620 y un arreglo de amplificadores de instrumentación.]{Comparación de la taza de error entre el AD620 y un arreglo de amplificadores de instrumentación\footnotemark}
                \label{fig:Amplificador_AD620_comparacion}
            \end{figure}
            \footnotetext{Muestra la diferencia en la taza de error por corriente suministrada de un amplificador de instrumentación AD620 y un arreglo de amplificadores de instrumentación convencional. Imagen tomada del Datasheet del AD620. Fuente: \url{https://www.analog.com/media/en/technical-documentation/data-sheets/AD620.pdf}}

            El componente cuenta con una entrada inversora (pin 2), una no inversora (pin 3), dos fuentes de alimentación (pin 4 y 7), una referencia o polo a tierra (pin 5), y una salida de la señal amplificada (pin 6). La ganancia del AD620 se puede ajustar mediante una resistencia externa conectada entre el pin 1 y el pin 8.
            \begin{figure}[H]
                \centering
                \includegraphics[width=0.6\textwidth]{img/Desarrollo/AD620_Diagrama.png}
                \caption[Esquema de conexión del amplificador de instrumentación AD620.]{Esquema de conexión del amplificador de instrumentación AD620\footnotemark}
                \label{fig:AD620_pinout}
            \end{figure}
            \footnotetext{Muestra el esquema de conexión del amplificador de instrumentación AD620, con sus terminales de alimentación, entrada y salida. Imagen tomada del Datasheet del AD620. Fuente: \url{https://www.analog.com/media/en/technical-documentation/data-sheets/AD620.pdf}}

            Resulta que, en el marco de operación de este amplificador de instrumentación, es posible determinar la ganancia saliente a partir de la resistencia conectada entre los pines 1 y 8 \cite{AD620_AnalogDevices}, mediante la ecuación~\ref{eq:ganancia}.

            \begin{equation}
                \label{eq:ganancia}
                G = 1 + \frac{49.4 k\Omega}{R_G}
            \end{equation}

            Donde $G$ es la ganancia del amplificador, $R_G$ es la resistencia conectada entre los pines 1 y 8 y la constante $49.4 k\Omega$ es el valor de la resistencia interna del AD620.


        \subsubsection{Filtro de Segundo Orden Sallen-Key Pasa Altas y Pasa Bajas.}
            Fue necesario implementar un circuito capaz de acondicionar las señales previamente adquiridas y amplificadas. Para ello, se utilizó un filtro activo Sallen-Key de segundo orden, reconocido por su simplicidad y su capacidad para ofrecer una respuesta de frecuencia precisa. Este filtro puede configurarse como pasa altas o pasa bajas, ajustando los valores de resistencias y capacitores según los requisitos específicos de la aplicación.

            \paragraph{Filtro Pasa Altas}
                Un filtro pasa altas permite el paso de frecuencias superiores a una frecuencia de corte, mientras que atenúa las frecuencias inferiores a esta.

                \begin{figure}[H]
                    \centering
                    \includegraphics[width=0.6\textwidth]{img/Desarrollo/Filtro_Pasa_Altas.png}
                    \caption[Diagrama de un filtro pasa altas de segundo orden Sallen-Key.]{Diagrama de un filtro pasa altas de segundo orden Sallen-Key\footnotemark}
                    \label{fig:Filtro_Pasa_Altas}
                \end{figure}
                \footnotetext{Muestra el diagrama de un filtro pasa altas de segundo orden Sallen-Key, con dos resistencias y dos capacitores.}


\newpage
\section{Desarrollo del Proyecto}
    \subsection{Desarrollo del circuito de ECG}
        \subsubsection{Amplificador de Instrumentación AD620}
            El AD620 es un amplificador diferencial de instrumentación de bajo costo y consumo, pero de una gran precisión y estabilidad de la compañía fabricante de semiconductores Analog Devices. Este amplificador es ideal para aplicaciones de bajo nivel de señal, como la adquisición de señales bioeléctricas, ya que ofrece una alta impedancia de entrada y una ganancia ajustable entre las terminales 1 y 8 del circuito integrado.

            Buscamos obtener una ganancia de 21, para esto se utilizó la ecuación~\ref{eq:ganancia} para calcular el valor de la resistencia $R_G$ que se conecta entre los pines 1 y 8 del AD620.

            \begin{equation}
                \label{eq:resistencia}
                R_G = \frac{49.4 k\Omega}{G - 1} = \frac{49.4 k\Omega}{21 - 1} = 2470 \Omega \approx 2.5 k\Omega
            \end{equation}

            La resistencia $R_G$ se aproximó a $2.5 k\Omega$ y se conectó entre los pines 1 y 8 del AD620 para obtener una ganancia aproximada a lo deseado como se muestra en la ecuación~\ref{eq:ganancia_teorica}.

            \begin{equation}
                \label{eq:ganancia_teorica}
                G = 1 + \frac{49.4 k\Omega}{2.5 k\Omega} = 20.76
            \end{equation}

        \subsubsection{Filtro de segundo orden Sallen-Key}
            Para la siguiente etapa del circuito, fue necesario implementar un tipo de circuito capaz de acondicionar las señales previamente adquiridas y amplificadas. Para ello, se utilizó un filtro activo Sallen-Key de segundo orden, reconocido por su simplicidad y su capacidad para ofrecer una respuesta de frecuencia precisa. Este filtro puede configurarse como pasa altas o pasa bajas, ajustando los valores de resistencias y capacitores según los requisitos específicos de la aplicación.

            Esta configuración fue seleccionada debido a que, al ser un filtro activo, no presenta ningún tipo de atenuación en la señal, a diferencia de un filtro pasivo, que sí atenúa las señales al manejar frecuencias bajas. Además, se aprovechan las características eléctricas de un amplificador operacional (op-amp), como la alta impedancia de entrada y la baja impedancia de salida, lo cual es beneficioso para evitar cargas en las etapas previas y posteriores al filtro.
        
            Existen tres tipos de aproximaciones para este tipo de filtro: Butterworth, Chebyshev y Bessel, cada una con sus propias características y aplicaciones. En este caso, se utilizó la aproximación de Butterworth, ya que presenta una respuesta de frecuencia más plana en la banda de paso, lo que significa que no hay sobrepicos en dicha banda y que la caída de la respuesta en la banda de rechazo es más suave. Esto contrasta con la aproximación de Chebyshev, que presenta sobrepicos en la banda de paso y una caída más abrupta en la banda de rechazo, o con la aproximación de Bessel, que ofrece una fase lineal.

            Al ser filtros activos, estos tienen la capacidad de amplificar la señal filtrada mediante una retroalimentación negativa en el op-amp, realizada a través de un divisor de tensión entre la terminal de salida y la entrada no inversora del op-amp, utilizando las resistencias denominadas $R_a$ y $R_b$.

            Dado que ya se amplificó la señal en la etapa anterior y que amplificar de esta manera podría introducir ruido al sistema, se optó por establecer una ganancia unitaria en el filtro, configurando $R_a$ como infinito y $R_b$ como cero.

            Para eliminar el ruido de baja frecuencia presente en la señal de ECG, se implementó un filtro pasa altas de segundo orden Sallen-Key. Este filtro permite el paso de frecuencias superiores a una frecuencia de corte, mientras que atenúa las frecuencias inferiores a esta.

            \paragraph{Filtro Activo Pasa Altos.}
                \subparagraph{Función de transferencia del filtro pasa altos.}
                    La función de transferencia de un filtro pasa altos de segundo orden Sallen-Key se puede calcular mediante la ecuación~\ref{eq:funcion_transferencia_pasa_altos}.

                    \begin{equation}
                        \label{eq:funcion_transferencia_pasa_altos}
                        \frac{v_o}{v_i}(S) = \frac{(1+\frac{R_b}{R_a}) \cdot s^2}{s^2 + s \cdot \frac{1}{C} \cdot (\frac{2}{R_2}- \frac{R_b}{R_1 \cdot R_a}) + \frac{1}{C^2 \cdot R_1 \cdot R_2}}
                    \end{equation}

                    Donde:
                    \begin{itemize}
                        \item $V_o$: Es la tensión de salida del filtro.
                        \item $V_i$: Es la tensión de entrada del filtro.
                        \item $S$: Es la variable compleja en el dominiode de Laplace.
                        \item $R_1$ y $R_2$: Son las resistencias del circuito del filtro.
                        \item $R_a$ y $R_b$: Son las resistencias asociadas a la ganancia del amplificador operacional.
                        \item $C$: Es la capacitancia del circuito del filtro.
                    \end{itemize}

                    En este diseño, se buscó mantener una ganancia unitaria en el filtro para evitar la amplificación adicional de la señal y minimizar la introducción de ruido. Por lo tanto, se estableció $R_a$ como infinito (circuito abierto) y $R_b$ como cero (corto circuito), simplificando la ganancia del amplificador operacional a la ecuación~\ref{eq:funcion_amplificador_operacional_pasa_altos}.

                    \begin{equation}
                        \label{eq:funcion_amplificador_operacional_pasa_altos}
                        A = 1 + \frac{R_b}{R_a} = 1
                    \end{equation}

                \subparagraph{Cálculo de factor m}
                    El factor $m$ es crucial para determinar la relación entre las resistencias $R_1$ y $R_2$ del filtro pasa altos. Se calcula mediante la ecuación~\ref{eq:factor_m_pasa_altos}.

                    \begin{equation}
                        \label{eq:factor_m_pasa_altos}
                        m = \frac{1+\sqrt{1+8Q^2(A-1)}}{4Q}
                    \end{equation}

                    Para un filtro Butterworth de segundo orden, que ofrece una respuesta en frecuencia plana en la banda de paso, el factor de calidad $Q$ es:

                    \begin{equation}
                        \label{eq:factor_calidad}
                        Q = \frac{1}{\sqrt{2}} = 0.7071
                    \end{equation}
                
                    Sustituyendo el valor de $A = 1$ y $Q = 0.7071$ en la ecuación~\ref{eq:factor_m_pasa_altos}, se obtiene el valor de $m$.

                    \begin{equation}
                        \label{eq:factor_m_pasa_altos_valor}
                        m = \frac{1+\sqrt{1+8(0.7071)^2(1-1)}}{4(0.7071)} \approx 0.7071
                    \end{equation}

                    Este valor de $m$ se utilizará para calcular las resistencias del filtro pasa altos.

                \subparagraph{Frecuencia de corte}
                    La frecuencia de corte $f_c$ se calcula con la ecuación~\ref{eq:frecuencia_corte_pasa_altos}.

                    \begin{equation}
                        \label{eq:frecuencia_corte_pasa_altos}
                        f_c = \frac{m}{2\pi \cdot k \cdot R_1 \cdot C}
                    \end{equation}

                    Donde:

                    \begin{itemize}
                        \item $f_c$: Frecuencia de corte deseada.
                        \item $k$: Factor que depende de la configuración del filtro (para sallen key estándar, $k = 1$).
                        \item $R_1$: Resistencia seleccionada.
                        \item $C$: Capacitancia seleccionada.
                    \end{itemize}

                    Seleccionamos valores prácticos y disponibles en el mercado para $R_1$ y $C$:
                    \begin{itemize}
                        \item $R_1 = 2.2 k\Omega$
                        \item $C = 3.3 \mu F$
                    \end{itemize}

                    Sustituyendo estos valores en la ecuación~\ref{eq:frecuencia_corte_pasa_altos}, se obtiene la frecuencia de corte del filtro pasa altos.

                    \begin{equation}
                        \label{eq:frecuencia_corte_pasa_altos_valor}
                        f_c = \frac{0.7071}{2\pi \cdot 1 \cdot 2.2 k\Omega \cdot 3.3 \mu F} \approx 15.5 Hz
                    \end{equation}

                    Esta frecuencia de corte es adecuada para eliminar componentes de baja frecuencia no deseadas, como el ruido de línea y otras interferencias presentes en señales ECG.

                \subparagraph{Selección de resistencias $R_2$}
                    La resistencia $R_2$ se calcula mediante la ecuación~\ref{eq:resistencia_R2_pasa_altos}.
                    \begin{equation}
                        \label{eq:resistencia_R2_pasa_altos}
                        R_2 = \frac{R_1}{m^2} = \frac{2.2 k\Omega}{(0.7071)^2} \approx 4.4 k\Omega 
                    \end{equation}

            \paragraph{Filtro Activo Pasa Bajos.}
                \subparagraph{Función de transferencia del filtro pasa bajos.}
                    La función de transferencia de un filtro pasa bajos de segundo orden Sallen-Key se puede calcular mediante la ecuación~\ref{eq:funcion_transferencia_pasa_bajos}.

                    \begin{equation}
                        \label{eq:funcion_transferencia_pasa_bajos}
                        \frac{v_o}{v_i}(S) = \frac{(1 + \frac{R_b}{R_a}) \cdot \frac{1}{R^2 \cdot C_1 \cdot C_2}}{s^2 + s \cdot \frac{1}{R} \cdot (\frac{2}{C_1} - \frac{R_b}{C_2 \cdot R_a}) + \frac{1}{R^2 \cdot C_1 \cdot C_2}}
                    \end{equation}

                    Donde:
                    \begin{itemize}
                        \item $V_o$: Es la tensión de salida del filtro.
                        \item $V_i$: Es la tensión de entrada del filtro.
                        \item $S$: Es la variable compleja en el dominiode de Laplace.
                        \item $R_a$ y $R_b$: Son las resistencias asociadas a la ganancia del amplificador operacional.
                        \item $R$: Es la resistencia del filtro.
                        \item $C_1$ y $C_2$: Son las capacitancias del circuito del filtro.
                    \end{itemize}
                    
                    En este diseño, al igual que con el filtro pasa altas, se busca mantener una ganancia unitaria para evitar amplificar la señal adicionalmente y minimizar la introducción de ruido. Por lo tanto, se establece $R_a = \infty$ (circuito abierto) y $R_b = 0$ (corto circuito), simplificando la ganancia del amplificador operacional a la ecuación~\ref{eq:funcion_amplificador_operacional_pasa_bajos}.

                    \begin{equation}
                        \label{eq:funcion_amplificador_operacional_pasa_bajos}
                        A = 1 + \frac{R_b}{R_a} = 1
                    \end{equation}

                \subparagraph{Cálculo de factor m}
                    El factor $m$ es esencial para determinar la relación entre las capacitancias $C_1$ y $C_2$ y asegurar que el filtro tenga las caracteríasticas de frecuencia deseadas. Se calcula mediante la ecuación~\ref{eq:factor_m_pasa_bajos}.

                    \begin{equation}
                        \label{eq:factor_m_pasa_bajos}
                        m = \frac{1+\sqrt{1+8Q^2(A-1)}}{4Q}
                    \end{equation}

                    Para un filtro Butterworth de segundo orden, que ofrece una respuesta en frecuencia plana en la banda de paso, el factor de calidad $Q$ es:

                    \begin{equation}
                        Q = \frac{1}{\sqrt{2}} = 0.7071
                    \end{equation}

                    Sustituyendo el valor de $A = 1$ y $Q = 0.7071$ en la ecuación~\ref{eq:factor_m_pasa_bajos}, se obtiene el valor de $m$.

                    \begin{equation}
                        \label{eq:factor_m_pasa_bajos_valor}
                        m = \frac{1+\sqrt{1+8(0.7071)^2(1-1)}}{4(0.7071)} = \frac{1 + \sqrt{1}}{2.8284}\approx 0.7071
                    \end{equation}

                    Este valor de m se utilizará para calcular las capacitancias del filtro.

                \subparagraph{Frecuencia de corte}
                    La frecuencia de corte del filtro se calcula con la siguiente ecuación~\ref{eq:frecuencia_corte_pasa_bajos}.

                    \begin{equation}
                        \label{eq:frecuencia_corte_pasa_bajos}
                        f_c = \frac{1}{2\pi \cdot k \cdot R \cdot m \cdot C_1}
                    \end{equation}

                    Donde:
                    \begin{itemize}
                        \item $f_c$: Frecuencia de corte deseada.
                        \item $k$: Factor que depende de la configuración del filtro (para sallen key estándar, $k = 1$).
                        \item $R$: Resistencia seleccionada.
                        \item $m$: Factor de ajuste.
                        \item $C_1$: Capacitancia seleccionada.
                    \end{itemize}

                    Se seleccionaron valores prácticos y disponibles en el mercado para $R$ y $C_1$:

                    \begin{itemize}
                        \item $R = 2.7 k\Omega$
                        \item $C_1 = 4.7 \mu F$
                    \end{itemize}

                    Sustituyendo estos valores en la ecuación~\ref{eq:frecuencia_corte_pasa_bajos}, se obtiene la frecuencia de corte del filtro pasa bajos.

                    \begin{equation}
                        \label{eq:frecuencia_corte_pasa_bajos_valor}
                        f_c = \frac{1}{2\pi \cdot 1 \cdot 2.7 k\Omega \cdot 0.7071\cdot 4.7 \mu F} = \frac{1}{2\pi \cdot 2.7 x10^3 \cdot 0.7071\cdot 4.7 x10^{-6}} \approx 17.737 Hz
                    \end{equation}

                    Esta frecuencia de corte es adecuada para limitar las frecuencias a aproximadamente 17.7Hz, filtrando el ruido de alta frecuencia y preservando la información relevante de la señal ECG.

                \subparagraph{Cálculo de la resistencia Ra}
                    La resistencia $R_a$ se calcula para establecer la ganancia deseada del amplificador operacional y se obtiene mediante la ecuación~\ref{eq:resistencia_Ra_pasa_bajos}.

                    \begin{equation}
                        \label{eq:resistencia_Ra_pasa_bajos}
                        R_a = \frac{2AR}{A-1}
                    \end{equation}

                    Sin embargo, dado que $A = 1$, el denominador se hace cero, lo que resulta en una resistencia infinita. 

                    \begin{equation}
                        \label{eq:resistencia_Ra_pasa_bajos_valor}
                        R_a = \frac{2 \cdot 1 \cdot 2.7 k\Omega}{1 - 1} = \frac{5.4 k\Omega}{0} = \infty
                    \end{equation}

                    Esto indica que $R_a$ debe ser infinita, es decir, se elimina del circuito (circuito abierto), confirmado que la ganancia del amplificador operacional es unitaria.

                \subparagraph{Cálculo de la capacitancia C2}
                    La capacitancia $C_2$ se relaciona con la capacitancia $C_1$ mediante el factor $m$ y se calcula mediante la ecuación~\ref{eq:capacitancia_C2_pasa_bajos}.

                    \begin{equation}
                        \label{eq:capacitancia_C2_pasa_bajos}
                        C_2 = m^2 \cdot C_1 = (0.7071)^2 \cdot 4.7 \mu F \approx 2.35 \mu F
                    \end{equation}

                    Para facilitar la implementación del circuito con componentes estándar, se seleccionó una capacitancia con un valor comercial cercano.

                    \begin{equation}
                        C_2 = 2.2 \mu F
                    \end{equation}