\setcounter{page}{1} % Iniciar la numeración de las páginas en este punto

\section{Introducción}


\section{Antecedentes}

    \subsubsection*{A chest-based continuous cuffless blood pressure method: Estimation and evaluation using multiple body sensors \cite{bodySensor}.}

    El artículo analiza el desarrollo de un dispositivo no invasivo para la estimación de la presión arterial (PA) utilizando sensores colocados en el pecho utilizando la bioimpedancia (BImp) como alternativa a la fotopletismografía (PPG) para la extracción del tiempo de llegada del pulso (PAT), se tomaron cinco lecturas de tiempo de pulso diferentes y se pudo estimar la PA sistólica y diastólica. 
    
    Se analizaron los datos de 41 participantes en diversas condiciones fisiológicas, incluidos los cambios de postura y los resultados mostraron que la combinación de PAT con la frecuencia cardíaca mejoró la precisión del cálculo de la PA, y las lecturas de PAT basadas en BImp fueron un 3\% más precisas que las basadas en PPG, lo que destaca el potencial de BImp para un control más eficaz de la PA.

    \subsubsection*{An Arterial Compliance Sensor for Cuffless Blood Pressure Estimation Based on Piezoelectric and Optical Signals \cite{piezoelectric}.}

    Este artículo propone el desarrollo de un pequeño sistema de monitoreo que integra una matriz de sensosres piezoeléctricos y un sensor óptico que monitorea las señales fisiológicas de la arteria radial. El sistema hace el cálculo del tiempo de tránsito del pulso (PTT) y correlaciona la ecuación de Moens-Korteweg con la velocidad de onda de pulso (PWV) para estimar la presión arterial sistólica (PAS) y diastólica (PAD).

    En un experimento con 20 participantes se compararon dos métodos de estimación de la presión arterial, el primero utilizando el modelo de regresión y el segundo modelo $P-\beta$ basado en la ecuación de Moens-Korteweg.
     

    %Se realizaron pruebas con 20 adultos y utilizando el modelo de regresión $P-\beta$  

    \subsubsection*{Development of IoT Based Cuffless Blood Pressure Measurement System \cite{Norsuriati_2021}.}

    El artículo presenta el desarrollo de un sistema IoT de medición de presión arterial sin uso de un baumanómetro, se propone un método basado en el tiempo de tránsito del pulso (PTT). Este método correlaciona el tiempo de retraso entre las señales fotopletismografía (PPG) registrada en la punta del dedo y el lóbulo de la oreja.

    Los datos son recolectados mediante un microcontrolador Arduino Uno y procesados con el software MATLAB para eliminar el ruido y obtener los picos de la señal PPG para calcular el PTT.
    
    Los resultados son mostrados en la aplicación ThinkSpeak y ThingView debido a que tiene la capacidad de almacenar y visualizar los datos en tiempo real. El error medio y la deviación estándar para la presión arterial sistólica (PAS) estimada es de $22,5 \pm 20,6$ mmHg y para la presión arterial diastólica (PAD) es de $1,6 \pm 1,2$ mmHg.

    \subsubsection*{Remote monitoring system of vital signs for triage and detection of anomalous patient states in the emergency room \cite{Moreno_2016}.}

    \subsubsection*{Diseño de un sistema internet de las cosas (IoT) para el monitoreo de la presión arterial \cite{Estrada_2021}. }

    En este artículo se presenta los procesos de diseño y construcción de un prototipo biomédico IoT para el monitoreo de la presión arterial de pacientes en su lugar de residencia. El sistema consta en colocar un brazalete en el brazo del paciente a la altura del corazón, el cual se le conecta una bomba de aire el cual infla el brazalete y un sensor de presión diferencial MPX5050DP, se realiza una conversión análoga-digital con ayuda de un microcontrolador y se envía los datos meidante una API REST al servidor web IoT ThinkSpeak donde se almacenan y se pueden visualizar en tiempo real.

    \subsubsection*{Development of Real-Time Cuffless Blood Pressure Measurement Systems with ECG Electrodes and a Microphone Using Pulse Transit Time (PTT) \cite{Electrodes_Microphone}.}

    El artículo presenta el desarrollo de un sistema de medición de presión arterial sin uso de un baumanómetro, se propone un método basado en el tiempo de tránsito del pulso (PTT). Este método correlaciona el tiempo de retraso entre las señales electrocardiográficas (ECG) y la señal de sonido del pulso. 

\section{Objetivos}
    \subsection{Objetivo General}
    Desarrollar un dispositivo para estimar la presión arterial de forma no invasiva utilizando sensores.
    \subsection{Objetivos Específicos}
    \begin{itemize}
        \item Diseñar un dispositivo utilizando sensores para la obtención de datos fisiológicos.
        \item Enviar los datos mediante Bluetooth a un dispositivo móvil.
        \item Implementar un algoritmo de procesamiento de señales para la estimación de la presión arterial.
        \item Visualizar los datos obtenidos en una aplicación android.
    \end{itemize}


